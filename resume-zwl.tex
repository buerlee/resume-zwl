% !TEX TS-program = xelatex
% !TEX encoding = UTF-8 Unicode
% !Mode:: "TeX:UTF-8"
\documentclass[14pt]{resume}
\usepackage{graphicx}
\usepackage{tabu}
\usepackage{multirow}
\usepackage{multicol}
\usepackage{progressbar}
\usepackage{zh_CN-Adobefonts_external}
\usepackage{linespacing_fix}
\usepackage{cite}

\begin{document}
\pagenumbering{gobble}

\begin{multicols}{4}
    \Large{
        \begin{tabu}{ r }
            \multirow{5}{1in}{
                \includegraphics[width=0.88in]{avatar}
            }
        \end{tabu}
    }
    \columnbreak
    \Large{
        \begin{tabu}{ l l }
            & \faBirthdayCake{1995.09.12} \\
            & \phone{(+86)17600535912} \\
            & \email{izhouwl@163.com} \\
            & \homepage[www.zhouweilin.cn]{https://zhouweilin.cn} \\
            & \github[github.com/Si3ver]{https://github.com/Si3ver} 
        \end{tabu}
    }
    \columnbreak
    \Large{
        \begin{tabu}{ r }
            \multirow{5}{3.5in}{
                \name{周伟林}
                \basicInfo{
                    \faSmileO{意向职位:web前端研发}
                }
            }
        \end{tabu}
    }
\end{multicols}

% 教育背景
\section{\faGraduationCap\  教育背景}
\datedsubsection{\textbf{北京邮电大学(硕士)\quad\quad\quad}{ 网络技术研究院 \quad\quad\quad }{ 计算机科学与技术 }}{2016.09 - 2019.06}
\datedsubsection{\textbf{北京工业大学(本科)\quad\quad\quad}{ 计算机学院     \quad\quad\quad\quad\quad}{ 信息安全 }}{2012.09 - 2016.06}

% 技能树
\section{\faCogs\ 技能树}

\begin{itemize}
    \item[\faTree] 熟悉HTML、CSS、JavaScript
    \item[\faTree] 熟悉Vue.js、ECharts库,了解Bootstrap、jQuery,Canvas、AntD等
    \item[\faTree] 熟悉Node.js,会配置Webpack,注重前端性能
    \item[\faTree] 熟悉python语言,了解数据可视化,会使用Matplotlib、Markdown、LaTex\footnote{本简历使用LaTex书写并编译}
    \item[\faTree] 熟悉网络协议,包括HTTP(S)、TCP/IP协议栈,了解运营商技术,有网络运维相关经验
    \item[\faTree] 无障碍阅读英文文档(CET-6),熟练使用Google、Stack Overflow检索
\end{itemize}

% 实习经历
\section{\faBriefcase\ 实习经历}

\datedsubsection{\textbf{美到家\quad\quad\quad\quad\quad\quad} \textbf{产品技术部\quad\quad\quad\quad}{ WEB前端研发}}{2018.11 - 如今}
\begin{itemize}
    \item[\faFlagO] 参与了BA端的页面开发,包括美妆试戴、智能测肤、美瞳试戴页面,展示客户端和魔镜端产生的埋点数据及智能导购。
    \item[\faCode] 使用的技术栈包括Vue,Muse UI,fetch,Echarts等。
    \item[\faCheck] 小公司扁平化管理,与CEO、CTO直接接触和交流,对个人成长很大。
\end{itemize}


\datedsubsection{\textbf{滴滴出行\quad\quad\quad\quad\quad} \textbf{质量技术部\quad\quad\quad\quad}{ WEB前端研发}}{2018.05 - 2018.09}
\begin{itemize}
    \item[\faFlagO] 参与了Omega项目的前端页面开发工作,包括编写客服反馈、安卓错误上报等页面。
    \item[\faCode] 使用自制模板引擎simplite与原生JS开发,并使用node打包编译上线。
    \item[\faCheck] 开发过程中,探究了模板引擎的内部原理,还熟悉了Echarts库的使用。
\end{itemize}

\datedsubsection{\textbf{西门子中国\quad\quad\quad\quad} \textbf{新闻传播部\quad\quad\quad\quad}{网站技术支持}}{2017.10 - 2018.03}
\begin{itemize}
    \item[\faFlagO] 工作包括CMS的网页、内容迁移,上传资源文件并返还cover id,整理每月流量报表,开发部分页面组件等工作。
    \item[\faCheck] 实习期间,主要锻炼了英文沟通和书面能力。
\end{itemize}

\datedsubsection{\textbf{北京江南天安科技\quad} \textbf{软件研发部\quad\quad\quad\quad}{Windows程序开发}}{2015.07 - 2015.09}
\begin{itemize}
    \item[\faFlagO] 参与了Ukey项目,完成了Ukey灌装秘钥程序windows程序开发的工作。
    \item[\faCheck] 主要锻炼了VS的调试能力。
\end{itemize}

% 项目经历
\section{\faUsers\ 项目经历}

\datedsubsection{\textbf{网络优化模型和算法\quad\quad\quad\quad\quad\quad}{设计、实现与仿真}}{2018.07 - 2017.09}
\begin{onehalfspacing}
\begin{itemize}
    \item[\faFlagO] 本项目属于实验室科研项目,研究内容为网络功能虚拟化场景下的VNF放置算法
    \item[\faFlagO] 经过调研发现,实际流量经常发生激增现象,导致丢包,会严重影响QoS
    \item[\faFlagO] 在VNFaaS部署到数据中心拓扑的前,可事先根据流量涨幅和突发概率的特征,据此建模来优化部署网络应用
    \item[\faCode] 在导师和实验室博士后师兄的指导下,完成了方案设计、代码实现和论文撰写
    \item[\faCheck] 用python实现了VNF放置算法的仿真环境NFVsimu,并在github开源
    \item[\faCheck] 抽象出了可扩展性的最优化模型,并证明其等价于二分匹配的子问题
    \item[\faCheck] 设计并实现了sVNFP和sVNFP-adv两个启发式算法,使丢包率降低了15\%-40\%
\end{itemize}
\end{onehalfspacing}

\datedsubsection{\textbf{移动端SPA应用:去旅行\quad\quad\quad\quad\quad\quad\quad\quad}{Webapp开发}}{2018.04 - 2018.06}
\begin{onehalfspacing}
\begin{itemize}
    \item[\faFlagO] 基于Vue技术栈,并使用了axios、better-scroll等开源库。
    \item[\faFlagO] 项目包含Header等公用组件,首页包含轮播图、Logo项目栏、热销推荐栏、周末游等组件,另外还开发了城市列表页、项目详情页、画廊页。
    \item[\faCode] 本地开发使用了mock数据,并使用docker和express搭建了一套简单的RESTful API。
    \item[\faCheck] 熟悉了Vue框架,丰富了移动端的开发经验和后台经验。
\end{itemize}
\end{onehalfspacing}

\datedsubsection{\textbf{移动web端:别踩白块儿\quad\quad\quad\quad\quad\quad\quad\quad}{移动端游戏开发}}{2018.01 - 2018.02}
\begin{onehalfspacing}
\begin{itemize}
    \item[\faCode] 使用canvas开发,数据层即4行4列矩阵,支持touch事件和响应式布局。
\end{itemize}
\end{onehalfspacing}

% 校园经历
\section{\faUniversity\ 校园经历}

\datedsubsection{\textbf{组织校园读书文化节活动\quad\quad\quad\quad\quad\quad\quad}{校学生会学拓部部员}}{2013.03 - 2013.05}
\begin{onehalfspacing}
\begin{itemize}
    \item[\faFlagO] 制作移动端web pages宣传页,用以收集同学书籍喜好和闲置书籍,组织书籍交换活动
    \item[\faFlagO] 组织趣味问答环节,答对问题数量越多,奖品越丰厚
    \item[\faFlagO] 组织队员制作路展物件,用印制板制作书形道具
    \item[\faFlagO] 路展过程中,收集同学们的书籍推荐词,用书签的方式贴到道具上
\end{itemize}
\end{onehalfspacing}

% 个人荣誉
\section{\faHeartO\ 个人荣誉}

\trophy [Weilin Zhou, Yuan Yang, Mingwei Xu, and Hao Chen, Accommodating dynamic traffic immediately: a VNF placement approach\footnote{ICC 2019评审中,CCF C类会议}]{https://github.com/Si3ver/SVNF}

\trophy [周伟林, 杨芫, 徐明伟. 网络功能虚拟化技术研究综述. 计算机研究与发展, 2018, 55(4): 675-688\footnote{国内计算机领域三大核心期刊之一,EI检索}]{http://crad.ict.ac.cn/CN/abstract/abstract3662.shtml}

\trophy [2015全国大学生网络技术大赛(思科网院杯),全国二等奖,排名第12]{http://www.catc.edu.cn/2015cup/news/19rr5gvqfnqk2.xhtml}

\trophy [北京邮电大学研究生一等学业奖学金、网研院优秀研究生]{xxx}

\trophy [北京工业大学学习优秀奖、科技创新奖、优秀毕业生、基础课奖学金,物理知识竞赛优等奖]{xxx}

% 特长爱好 + 自我评价
\section{\faInfo\ 关于我}

\faTerminal 研究生期间研究网络,课外接触到前端,激发了自己对前端的兴趣。喜欢前端开发过程中及时反馈的特点,特别喜欢灵活的JavaScript语言,丰富的nodejs社区。

\faBook 平时也喜欢阅读、看综艺和电影,阅读了部分文学名著,对经济学、心理学也有部分了解。

\faBicycle 喜欢户外运动,曾徒步登泰山、恒山,慕田峪、箭扣、墙子路。与小伙伴一天内骑车160公里从京城到京郊游玩

\faHeartbeat 坚持游泳,偶尔也打打羽毛球、乒乓球

\end{document}
